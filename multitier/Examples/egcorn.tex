\begin{flushleft}\Large\bf A Two-Phase Corn Seed Germination Experiment
\end{flushleft}

(Prof.~T.B.~Bailey~Jr., Iowa State University, kindly provided 
a more complicated version of this experiment.)
A study to investigate corn seed germination involved a field and a
laboratory phase.  In the field phase, an experiment to produce corn seed
was conducted at three sites; at each site a randomized complete block
design of two blocks was run to investigate the differences between three
mechanical harvesters.  Four samples of seed were harvested from each plot,
%and these 
combined %.  The seed from each of the 18 plots was 
and then divided into 36 lots.  In the laboratory phase of the 
experiment, there were nine
containers each with four plates that were to be used for germinating seed
in each of 18 intervals.  In each of these intervals, the 36 lots of the
seed from a plot were placed on the four plates in the nine
containers, the interval in which the seed from a plot was germinated
being assigned at random.  In
each interval, nine temperature-moisture conditions, referred to as nine
treatments, were randomly assigned to the nine containers.  
Thus the inherent crossing of containers with intervals was ignored.
The percentage germination of the seeds was recorded for each
plate.

This two-phase experiment differs from others in
that treatments are introduced in both the first and second phases.  The
sets for this experiment are plates, lots, harvesters and treatments and
the tiers are 
${\cal F}_{\rm plates} = \setof{\mbox{Intervals}, 
\mbox{Containers}, \mbox{Plates}}$, ${\cal F}_{\rm lots} = 
\setof{\mbox{Sites}, \mbox{Plots}, \mbox{Blocks}, \mbox{Lots}}$, 
${\cal F}_{\rm harvesters} = \setof{\mbox{Harvesters}}$ and 
${\cal F}_{\rm treatments} = \setof{\mbox{Temperatures},
\mbox{Moistures}}$.  There are three randomizations: harvesters to plots, 
lots to plates and treatments to plates.  The first two randomizations are 
composed and the third is partly coincident with and partly independent of 
the second. Figure~\ref{fig:corn} gives the randomization diagram. New in this 
diagram is the black circle preceded by an arrow and with lines leading from 
it to both Containers and Plates --- it indicates that Lots are randomized to 
$\mbox{Containers}\wedge\mbox{Plates}$. This randomization and that of 
$\mbox{Plots}\wedge\mbox{Sites}\wedge\mbox{Blocks}$ are independent, 
but can be reduced to single randomization. The randomization of Lots 
is coincident with that of treatments. On the other hand the randomization of 
$\mbox{Plots}\wedge\mbox{Sites}\wedge\mbox{Blocks}$ and treatments are 
independent, but cannot be reduced to a single randomization.

\begin{figure}[htbp]
\centering
\begin{picture}(12,6)
\put(-1,4.05){\begin{tierbox}3 & Harvesters\end{tierbox}}
\put(1.85,4.15){\vector(1,0){1.6}}
\put(3,4){\begin{tierbox}3 & Sites\\2 & Blocks in S\\
3 & Plots in S$\wedge$B\\& \\36 & Lots in S$\wedge$B$\wedge$P\end{tierbox}}
%combinations on left
\put(7.8,4.6){\blob}
\put(7.8,4.6){\line(-2,1){1.2}}
\put(7.8,4.6){\line(-1,0){1.2}}
\put(7.8,4.6){\line(-3,-1){1.2}}
\put(7.8,4.6){\vector(2,-1){1.35}}
%combinations on right
\put(6.9,3.1){\vector(1,0){0.9}}
\put(7.9,3.1){\blob}
\put(7.9,3.1){\line(4,1){1.25}}
\put(7.9,3.1){\line(4,-1){1.25}}
%\put(8.1,3.6){\makebox(1,0)[l]{$4$ \nlevnamesep $L_1$}}
%\put(8.1,3){\makebox(1,0)[l]{$9$ \nlevnamesep $L_2$}}}
%\put(9.3,3.6){\vector(1,0){0.8}}
%\put(9.3,3){\vector(1,0){0.8}}
\put(8.7,3.2){\begin{tierbox}18 & Intervals\\
4 & Plates in I$\wedge$C\\9 & Containers in I\end{tierbox}}
\put(3.4,1.05){\begin{tierbox}3 & Temperatures\\3 & Moistures\end{tierbox}}
%\put(8,1){$\biggr\rbrace$}
%combinations on left
\put(7.8,1.15){\blob}
\put(7.8,1.15){\line(-4,1){1}}
\put(7.8,1.15){\line(-4,-1){1}}
\put(7.8,1.15){\vector(1,1){1.45}}
\end{picture}
\caption{Composed and coincident randomizations in the two-phase corn experiment}
\label{fig:corn}
\end{figure}

The permutation group used for the first randomization is $S_3\wr S_2
\wr S_3$ (since Lots are only implicit at this stage) while that for
the coincident second and third randomizations is $S_4 \wr S_9\wr S_{18}$.

The factor Lots could be omitted from the above description, but it is
included to emphasize that the $36$~lots from a given plot must not be
allocated systematically to the $36$~plates. 
