\begin{flushleft}\Large\bf Superimposed Experiment using a Row-and-Column Design
\end{flushleft}

\nocite{Freeman59}Free\-man (1959) describes a cherry experiment in which a
large number of rootstocks had been tested using a randomized complete
block design for 20 years.  The trees from 10 rootstocks in three blocks
were than to be used for investigating five virus treatments.  The five
virus treatments were assigned using the extended Youden square 
shown in table~\ref{t:SuperBIBD},  the rootstocks corresponding to the
columns and the blocks to the rows.

\begin{table}[htbp]
\begin{center}\begin{tabular}{lr|*{10}{c}}
 &  & \multicolumn{10}{c}{Rootstocks} \\
%\hline \\
       &     & 1 & 2 & 3 & 4 & 5 & 6 & 7 & 8 & 9 & 10 \\
\hline
       &   I & A & B & A & C & D & C & B & E & E & D  \\
Blocks &  II & D & E & B & D & E & A & C & C & A & B  \\
       & III & E & A & C & E & B & D & D & B & C & A  \\
%\hline \\
\end{tabular}\end{center}
\caption{\label{t:SuperBIBD}Virus Treatment for each Block-Rootstock 
combination}
\end{table}

The sets for this experiment are trees, rootstocks and treatments and the
tiers are ${\cal F}_{\rm trees} = \setof{\mbox{Blocks}, \mbox{Trees}}$,
${\cal F}_{\rm rootstocks} = \setof{\mbox{Rootstocks}}$ and ${\cal
F}_{\rm treatments} = \setof{\mbox{Viruses}}$.  There are two
randomizations: rootstocks to trees in the initial experiment and virus
treatments to trees, taking into account the rootstocks, in the revised
experiment.  In this experiment, the two randomizations are u-inclusive in
that the unrandomized factors for the second randomization include factors 
from both tiers, ${\cal F}_{\rm trees}$ and ${\cal F}_{\rm rootstocks}$, of 
the first randomization.

In Figure~\ref{fig:cherry} the dashed oval once again shows the new
pseudotier created by the first randomization.
\begin{figure}[htbp]
\centering
\begin{picture}(12,4)
\put(0,2.5){\begin{tierbox}5 & Viruses\end{tierbox}}
\put(2.25,2.57){\vector(4,-1){0.665}}
\put(2.9,2.26){$\Diamond$}
\put(3.2,2.4){\line(1,0){5.25}}
\put(3.2,2.4){\line(2,-3){1.15}}
\put(4,0.5){\begin{tierbox}10 & Rootstocks\end{tierbox}}
\put(7.15,0.65){\vector(1,1){1.25}}
\put(8,2){\begin{tierbox}3 & Blocks\\10 & Trees in B\end{tierbox}}
\curvedashes[4pt]{0,1,1}
\closecurve(4.5,0,11,0,12,1,12,2,11,3,4.5,3,3.5,2,3.5,1)
\end{picture}
\caption{Unrandomized-inclusive randomizations in the 
superimposed experiment using a row-and-column design}
\label{fig:cherry}
\end{figure}

Now the diamond after the arrow indicates that viruses are randomized
to those combinations of Block and Rootstock that occur as a result of
the first randomization. The group for the first randomization is
$S_{10}\wr S_3$, while that for the second is $S_{10} \times S_3$,
which is a subgroup of the former. In contrast to
wheat variety experiment, both the group and the systematic plan for the
second randomization are constrained by the result of the first. This
double constraint is reflected in the number of plans which the
designer must write out. If the result of the first randomization is
 shown in Table~\ref{t:roots} and the constrained systematic plan
in Table~\ref{t:SuperBIBD} is randomized by $S_{10} \times S_3$ to the
plan in Table~\ref{t:virus} then the designer will also have to write
out Table~\ref{t:orchard} for the orchard worker to follow.

\begin{table}[htbp]
\begin{center}\begin{tabular}{lr|*{10}r}
 &  & \multicolumn{10}{c}{Trees} \\
%\hline \\
       &     & 1 & 2 & 3 & 4 & 5 & 6 & 7 & 8 & 9 & 10 \\
\hline
       &   I & 7 & 2 & 1 & 10 & 6 & 3 & 5 & 9 & 4 & 8\\
Blocks &  II & 5 & 6 & 3 & 4 & 9 & 1 & 8 & 7 & 2 & 10\\
       & III & 8 & 6 & 3 & 9 & 7 & 4 & 2 & 10 & 1 & 5
%\hline \\
\end{tabular}\end{center}
\caption{Allocation of rootstocks after the first randomization}
\label{t:roots}
\end{table}

\begin{table}[htbp]
\begin{center}\begin{tabular}{lr|*{10}{c}}
 &  & \multicolumn{10}{c}{Rootstocks} \\
%\hline \\
       &     & 1 & 2 & 3 & 4 & 5 & 6 & 7 & 8 & 9 & 10 \\
\hline
       &   I & C & D & B & E & A & D & B & E & A & C\\
Blocks &  II & D & B & A & B & E & A & D & C & C & E\\
       & III & A & E & E & C & D & B & C & A & B & D
%\hline \\
\end{tabular}\end{center}
\caption{Allocation of viruses after the second randomization}
\label{t:virus}
\end{table}

\begin{table}[htbp]
\begin{center}\begin{tabular}{lr|*{10}r}
 &  & \multicolumn{10}{c}{Trees} \\
%\hline \\
       &     & 1 & 2 & 3 & 4 & 5 & 6 & 7 & 8 & 9 & 10 \\
\hline
       &   I & 7B & 2D & 1C & 10C & 6D & 3B & 5A & 9A & 4E & 8E\\
Blocks &  II & 5E & 6A & 3A & 4B & 9C & 1D & 8C & 7D & 2B & 10E\\
       & III & 8A & 6B & 3E & 9B & 7C & 4C & 2E & 10D & 1A & 5D
%\hline \\
\end{tabular}\end{center}
\caption{Plan for the orchard worker}
\label{t:orchard}
\end{table}

The structure formulae for the superimposed experiment are 
\[\begin{array}{c}
  \mbox{3 Blocks} \nest \mbox{10 Trees} \\
  \mbox{10 Rootstocks}  \\
  \mbox{5 Viruses}.
\end{array}\]

The Hasse diagrams for this experiment are trivial as is the matrix of 
efficiencies for the structure on rootstocks in relation to that on trees. 
The matrix of efficiencies for the structure on treatments in relation to 
the joint decomposition of trees and rootstocks is given in 
Table~\ref{tab:DecompSuper}. Note that for this example, we have that 
Viruses is orthogonal with respect to the 
structure on trees since
\begin{equation*}\begin{split}
\lambda_{\mbox{Plots} \nesting{\mbox{Blocks}},\:\mbox{Viruses}} 
  &= \lambda_{\mbox{Plots} \nesting{\mbox{Blocks}}
                            \sresid\mbox{Rootstocks},\:\mbox{Viruses}} \\
  & \qquad+ \lambda_{\mbox{Plots} \nesting{\mbox{Blocks}}
                           \scombine\mbox{Rootstocks},\:\mbox{Viruses}} \\
  & = \dfrac{1}{6} + \dfrac{5}{6} \\
  & = 1.
\end{split}\end{equation*}

\begin{table}[htbp]
\renewcommand{\arraystretch}{1.9}
\[
\begin{array}{cc}
&
\begin{array}{cc} 
\mbox{Mean} & \mbox{Viruses}
\end{array}
\\
\begin{array}{l}
\mbox{Mean}\\
\mbox{Blocks}\\
\mbox{Plots} \nesting{\mbox{Blocks}}\\
\quad\combine\:\mbox{Rootstocks} \vphantom{\dfrac{1}{6}}\\
\quad\resid\:\mbox{Rootstocks} \vphantom{\dfrac{5}{6}}\\
\end{array}
&
\left[
\begin{array}{cc}
 \fish{Mean}{1} & \fish{Viruses}{0} \\
 0 & 0\\
 \\
 0 & \dfrac{1}{6}\\
 0 & \dfrac{5}{6}\\
\end{array}
\right]
\end{array}
\]
\caption{Matrix of efficiency factors 
   $\Lambda_{{\cal P} \protect\scombine {\cal Q}, {\cal R}}$ 
   for the superimposed experiment}
\label{tab:DecompSuper}
\end{table}

The analysis of variance table derived from this structure is given in 
Table~\ref{tab:ANOVASuper}. It shows that Viruses is confounded with 
both Rootstocks and that part of  $\mbox{Plots} \nesting{\mbox{Blocks}}$ 
that is orthogonal to Rootstocks. A consequence of this is that 
four Rootstocks degrees of freedom are not able to be separated from 
Virus differences. However, there is five Rootstocks degrees of freedom 
that are free of Virus differences and this analysis provides the 
corresponding sums of squares.  Further, we note that while Viruses 
is balanced with respect to Rootstocks, the reverse is not true. 
The aspect of this experiment that the analysis elucidates is that, 
in the second randomization, Rootstocks has acted as a `block' factor 
in that it has been involved in restrictions that were placed on this 
randomization.

\begin{table}[htbp]
%\renewcommand{\arraystretch}{1.9}
\begin{center}
\begin{tabular}{lrrrc}
Source & \multicolumn {3}{c}{DF} & Efficiency factor \\
\hline
Blocks & 2 \\
$\mbox{Plots} \nesting{\mbox{Blocks}}$ & 27\\
\quad Rootstocks & & 9          \\
\qquad Viruses   & & & 4 & 1/6  \\
\qquad Residual  & & & 5        \\
\quad Residual   & & 18         \\
\qquad Viruses   & & & 4 & 5/6  \\
\qquad Residual  & & & 14       \\
\hline
Total            & 29
\end{tabular}
\end{center}
\caption{Analysis variance table for the superimposed experiment}
\label{tab:ANOVASuper}
\end{table}