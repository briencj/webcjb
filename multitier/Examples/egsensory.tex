\begin{flushleft}\Large\bf A Two-Phase Sensory Experiment
\end{flushleft}

\nocite{Brien83}Brien (1983) describes a simple, sensory experiment to 
evaluate a set of wines made from the produce of a field trial in order to 
test the effects of several viticultural treatments.  
In the field trial, the treatments are assigned to plots 
according to a randomized complete block design. Then the produce from each 
plot is separately made into wine which is evaluated 
at a tasting inwhich several judges are given the wines over a number of 
sittings.  One wine is presented for scoring to each judge at a sitting and 
each wine is presented only once to a judge.  The order of presentation of 
the wines is randomized for each judge.  This experiment is then a two-phase 
experiment \nocite{McIntyre55}(McIntyre, 1955).  In the first, or field, phase 
the field trial is conducted, and in the second, or evaluation, phase the wine 
made from the produce of each plot in the field trial is evaluated by several 
judges.

The sets for this experiment are evaluations, wines and treatments,
 and the
tiers are 
${\cal F}_{\rm evaluations} = \setof{\mbox{Judges}, \mbox{Sittings}}$, 
${\cal F}_{\rm wines} = \setof{\mbox{Blocks}, \mbox{Plots}}$ and 
${\cal F}_{\rm treatments} = \setof{\mbox{Treatments}}$.  There are two
randomizations --- treatments to wines and wines to evaluations --- and
they are composed.  The crucial aspect of these two randomizations is that 
no account is taken of the randomized factors from the first phase, 
${\cal F}_{\rm treatments}$, when doing the randomization of the second 
phase; the only factors explicitly included in the second randomization are 
those from ${\cal F}_{\rm wines}$ and ${\cal F}_{\rm evaluations}$. 
However, the two randomizations taken together have the effect of 
randomizing treatments onto evaluations.
 
The randomization diagram in Figure~\ref{fig:wine} summarizes 
the randomization. 
Although Judges and Sittings are inherently crossed
on the set of evaluations, the experimenter chose to ignore this
inherent structure and randomize according to the group $S_{bt} \wr
S_j$.

\begin{figure}[htbp]
\centering
\begin{picture}(12,2)(0.5,0)
\put(0,0.5){\begin{tierbox}$t$ & Treatments\end{tierbox}}
\put(4.5,0.75){\begin{tierbox}$b$ & Blocks\\$t$ & Plots in B\end{tierbox}}
\put(8.7,0.75){\begin{tierbox}$j$&Judges\\$bt$&Sittings in J\end{tierbox}}
\put(3,0.6){\vector(1,0){1.9}}
\put(7.25,0.6){\vector(1,0){1.85}}
\put(8,0.6){\blob}
\put(8,0.6){\line(-3,2){0.75}}
\end{picture}
\caption{Composed randomizations in the two-phase sensory experiemnt}
\label{fig:wine}
\end{figure}

A more complicated two-phase sensory experiment is described by 
\nocite{Brien99}Brien and Payne (1999).
