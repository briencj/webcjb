\begin{flushleft}\Large\bf A Two-Phase Wheat Variety Trial
\end{flushleft}

(This example is based on a problem described to us by
Kathy Haskard, BiometricsSA.)
In the field phase of this experiment 49 lines of wheat are investigated 
using a randomized complete block design with four blocks. 
The produce of each plot is to be analysed using a gas chromatograph 
in which just seven samples can be processed in a single run.  
A $7 \times 7$ balanced lattice square design 
with four replicates is used to assign the blocks, plots and lines to 
four intervals in each of which there are seven runs at which samples are 
processed at seven consecutive times. 

The sets for this experiment are analyses, plots and lines and the tiers are 
${\cal F}_{\rm analyses} = \setof{\mbox{Intervals}, \mbox{Runs}, 
                                  \mbox{Times}}$, 
${\cal F}_{\rm plots} = \setof{\mbox{Blocks}, \mbox{Plots}}$, 
and ${\cal F}_{\rm lines} = \setof{\mbox{Lines}}$.  
There are two randomizations: lines to plots and plots to analyses. 

There are two aspects of this experiment that prevent the use of 
two randomizations that are composed:  (i) lines are randomized to plots 
within blocks in the first randomization;  (ii) plots within blocks 
are to be randomized in the second randomization 
to the levels of, not a single analyses factor,
but two different analyses factors, and their combinations.
%to the levels combinations of, not a single set of analyses factors,
%but three sets of analyses factors.
%not so that they are confounded with a single analyses term, 
%but so that they are confounded with three of these terms; and, 
These obstacles could be overcome, and composed randomizations employed, 
by assigning the 49 plots within a block completely at random to the 
49 run-time combinations within an interval. However, when it comes to 
the analysis of variance, it would not be feasible to isolate 
Runs within Intervals and Times within Intervals effects, 
Lines being hopelessly confounded with these terms. So we need to 
choose different partitions of Plots within Blocks to randomize to 
different sets of analyses factors. A partition into 7 parts of 
Plots within Blocks will be randomized to Runs within Intervals and 
another partition into 7 parts to Times within Intervals 
so that all 49 Plots within a block are randomized 
to the 49 Runs by Times combinations within an interval. A pseudofactor with
7 levels and nested within blocks, say $P_1$, is set up; it indexes which 
plots are randomized to which runs. Similary, a second pseudofactor, 
say $P_2$, is set up for plots randomized to times. 
%Indeed, $P_R$ and $P_T$ are the same factors as Runs and Times, respectively, 
%except that they are nested within Blocks.
%and the remainder with the Runs by Times combinations within Intervals. 
The selection of partitions of Plots within Blocks needs to be 
done so that, in the analysis of variance, Lines are balanced with the 
three terms involving Runs and Times. Clearly, to choose these 
partitions it is necessary to know which Lines are associated with 
which Plots within Blocks --- that is, to know the outcome of the 
first randomization. In this example, a balanced lattice square design to 
specify that the plots with certain lines should be processed at the 
same run in a particular interval and at the same time in a particular 
interval. This divison of the lines into groups for randomizing in a 
particular interval, simultaneously results in a division into groups of 
the plots in the block assigned to that interval. Again a pair of 
Lines pseudofactors for each interval can be set up, one to indicate those 
lines that occur in the same run in that interval and the other to 
indicate those that occur at the same time in that interval. Suppose the pairs
are labelled: $(L_1,L_2)$, $(L_3,L_4)$, $(L_5,L_6)$ and $(L_7,L_8)$. 

The randomization diagram for this experiment is given
in Figure~\ref{fig:wheat}. It has two new features. 
The dashed oval shows the pseudotier created by the first randomization. 
One diamond, with lines going towards it, indicates that four of the \cb[24]
Lines pseudofactors and the 
correspondng Plots pseudofactor are taken in combination, but
that not all combinations of the Lines pseudofactors with the Plots 
pseudofactor occurs: those that do are randomized to Runs or Times 
within Intervals, as appropriate. 
The diamond shows combinations of levels of 
factors from different tiers, in contrast to a black circle that shows 
combintions from the same tier.

\begin{figure}[htbp]
\centering
\begin{picture}(13.5,4)
\curvedashes[4pt]{0,1,1}
\closecurve(1,0, 8.7,0, 9.7,1, 9.7,2, 8.7,3, 1,3, 0,2, 0,1)
\put(0.1,1.5){\begin{tierbox}49 & Lines\hspace{250pt}\end{tierbox}}
\put(2.95,1.6){\vector(1,0){1.9}}
\put(2.2,1.6){\line(1,0){0.5}}
\put(2.6,1.6){\blob}
\put(2.6,1.6){\line(2,3){0.8}}
\put(2.6,1.6){\line(2,-3){0.8}}
\put(3.6,2.8){\makebox(1,0)[l]{$7 \hspace{\nlevnamesep} L_1,L_3,L_5,L_7$}}
\put(3.6,0.3){\makebox(1,0)[l]{$7 \hspace{\nlevnamesep} L_2,L_4,L_6,L_8$}}
\put(4.5,1.7){\begin{tierbox}4 & Blocks\\49 & Plots in B\end{tierbox}}
\put(6.9,2.05){\vector(3,1){4.05}}
\put(7.9,1.6){\line(-1,0){0.4}}
\put(7.9,1.6){\blob}
\put(7.9,1.6){\line(3,2){0.5}}
\put(7.9,1.6){\line(3,-2){0.5}}
\put(8.5,2){\makebox(1,0)[l]{$7 \hspace{\nlevnamesep} P_1$}}
\put(8.5,1.2){\makebox(1,0)[l]{$7 \hspace{\nlevnamesep} P_2$}}
\put(10,2.85){\line(-1,0){3.5}}
\put(10,2.85){\line(-1,-1){0.65}}
\put(10,0.3){\line(-1,0){3.5}}
\put(10,0.3){\line(-1,1){0.65}}
\put(9.975,2.7){$\Diamond$}
\put(9.975,0.15){$\Diamond$}
\put(10.275,2.85){\vector(1,0){0.65}}
\put(10.275,0.3){\vector(1,0){0.65}}
\put(10.5,1.7){\begin{tierbox}4 & Intervals\\7 & Runs in I\\&\\&\\&\\&\\%&\\
7 & Times in I\end{tierbox}}
\end{picture}
\caption{Randomized-inclusive randomizations in the wheat experiment}
\label{fig:wheat}
\end{figure}

The permutation group in the first randomization is $S_{49}\wr
S_4$ on plots. In the second randomization it is $(S_7 \times S_7) \wr
S_4$ on analyses. Thus the effect of the first randomization is 
not to constrain the second randomization but to constrain the 
choice of initial systematic plan allocating the diamond combinations
to analyses.

Note that, like the sensory and the corn-seed germination experiment, 
this is 
a two-phase experiment and involves different units in its field and 
laboratory phases. However, it differs from the other examples in that 
it involves r-inclusive, rather than composed, 
multiple randomizations. That is, factors from both tiers of the first 
randomization, ${\cal F}_{\rm plots}$ and ${\cal F}_{\rm lines}$, are 
explicitly included in the randomized factors for the 
second randomization.

The sets for this experiment are analyses, plots and lines. 
Because pseudofactors were introduced for lines and plots, the sets 
of factors on which 
the structure for the analysis will be based are 
${\cal F}_{\rm analyses} 
     = \setof{\mbox{Intervals}, \mbox{Runs}, \mbox{Times}}$, 
${\cal F}_{\rm plots} \cup {\cal H}_{\rm plots} 
     = \setof{\mbox{Blocks}, \mbox{Plots}, \mbox{P}_1, \mbox{P}_2}$, 
and ${\cal F}_{\rm lines} \cup {\cal H}_{\rm lines}
     = \setof{\mbox{Lines}, \mbox{L}_1, \mbox{L}_2, \mbox{L}_3, \mbox{L}_4, 
                            \mbox{L}_5, \mbox{L}_6, \mbox{L}_7, \mbox{L}_8}$.  
The structure formulae derived from these sets are 
\[\begin{array}{c}
  \mbox{4 Intervals} \nest (\mbox{7 Runs} \cross \mbox{7 Times}) \\
  \mbox{4 Blocks} \nest (\mbox{49 Plots} \pseudo (\mbox{7 P}_1 + \mbox{7 P}_2))  \\
  \mbox{49 Lines} \pseudo (\mbox{7 L}_1 + \mbox{7 L}_2 + \mbox{7 L}_3 
      + \mbox{7 L}_4 + \mbox{7 L}_5 + \mbox{7 L}_6 + \mbox{7 L}_7 + \mbox{7 L}_8).
\end{array}\]
and the Hasse diagrams for the structures derived from them are given in 
Figure~\ref{f:HasseWheat}. It is with respect to the structures summarized 
in these figures that the design is structurally balanced. 
The non-zero efficiency factors are
%\[\begin{array}{rll}
%   \lambda_{\mbox{Intervals},\:\mbox{Blocks}} &= 1                   \\
%   \lambda_{\mbox{Runs} \nesting{\mbox{I}},\:\mbox{P}_1 \nesting{\mbox{B}}} &=
%   \lambda_{\mbox{Times} \nesting{\mbox{I}},\:\mbox{P}_2 \nesting{\mbox{B}}}= 1\\
%   \lambda_{\mbox{P}_1 \nesting{\mbox{B}},\:\mbox{L}_i} &=
%   \lambda_{\mbox{P}_2 \nesting{\mbox{B}},\:\mbox{L}_j} = 0.25 
%                                  & i=1,3,5,7; j=2,4,6,8             \\
%   \lambda_{\mbox{Plots} \nesting{\mbox{B}},\:\mbox{L}_k} &= 0.75 
%                                  & k=1\ldots 8
%\end{array}\]
\begin{alignat*}{5}
   \lambda_{\mbox{Intervals},\:\mbox{Blocks}} & = 1                  \\
   \lambda_{\mbox{Runs} \nesting{\mbox{I}},\:\mbox{P}_1 \nesting{\mbox{B}}} & =
   \lambda_{\mbox{Times} \nesting{\mbox{I}},\:\mbox{P}_2 \nesting{\mbox{B}}} 
                                                        = 1         \\
   \lambda_{\mbox{P}_1 \nesting{\mbox{B}},\:\mbox{L}_i} & =
   \lambda_{\mbox{P}_2 \nesting{\mbox{B}},\:\mbox{L}_j} = 0.25
                                   & \quad i&=1,3,5,7; j=2,4,6,8      \\
   \lambda_{\mbox{Plots} \nesting{\mbox{B}},\:\mbox{L}_k} &= 0.75   
                                  & \quad k&=1\ldots 8
\end{alignat*}

The analysis of variance table for this example is given in 
Table~\ref{tab:ANOVAWheat}.

\begin{figure}[htbp]
\begin{minipage}{0.45\columnwidth}
\centering
\begin{picture}(4,5.5)(-2,-1)
\put(0,3){\blob}
\thook{0,3.3}{Universe $1$, $1$}
\put(0,2){\blob}
\rrhook{0.3,2.25}{\begin{tabular}{@{}c@{}}
     Intervals\\4, 3 \end{tabular}}
\put(-1,1){\blob}
\llhook{-1.3,1.25}{\begin{tabular}{@{}c@{}}
     $\mbox{Runs} \nesting{\mbox{I}}$\\28, 24 \end{tabular}}
\put(1,1){\blob}
\rrhook{1.3,1.25}{\begin{tabular}{@{}c@{}}
     $\mbox{Times} \nesting{\mbox{I}}$\\28, 24 \end{tabular}}
\put(0,0){\blob}
\bhook{0,-0.3}{\begin{tabular}{@{}c@{}}
     $\mbox{R} \inter \mbox{T} \nesting{\mbox{I}}$\\196, 144 \end{tabular}}
\multiput(0,0)(-1,1){2}{\line(1,1){1}}
\multiput(0,0)(1,1){2}{\line(-1,1){1}}
\put(0,2){\line(0,1){1}}
\end{picture}
\end{minipage}
\hfill
\begin{minipage}{0.45\columnwidth}
\centering
\begin{picture}(4,5.5)(-2,-1)
\put(0,3){\blob}
\thook{0,3.3}{Universe $1$, $1$}
\put(0,2){\blob}
\rrhook{0.3,2.25}{\begin{tabular}{@{}c@{}}
     Blocks\\4, 3 \end{tabular}}
\put(-1,1){\blob}
\llhook{-1.3,1.25}{\begin{tabular}{@{}c@{}}
     $\mbox{P}_1 \nesting{\mbox{B}}$\\28, 24 \end{tabular}}
\put(1,1){\blob}
\rrhook{1.3,1.25}{\begin{tabular}{@{}c@{}}
     $\mbox{P}_2 \nesting{\mbox{B}}$\\28, 24 \end{tabular}}
\put(0,0){\blob}
\bhook{0,-0.3}{\begin{tabular}{@{}c@{}}
     $\mbox{Plots} \nesting{\mbox{B}}$\\196, 144 \end{tabular}}
\multiput(0,0)(-1,1){2}{\line(1,1){1}}
\multiput(0,0)(1,1){2}{\line(-1,1){1}}
\put(0,2){\line(0,1){1}}
\end{picture}
\end{minipage}

\begin{center}
\begin{minipage}{0.75\columnwidth}
\centering
\begin{picture}(9,6.5)(-4.25,-1)
\put(0,4){\blob}
\thook{0,4.3}{Universe $1$, $1$}
\multiput(-4,2)(1,0){4}{\blob}
\llhook{-3.9,2.25}{\begin{tabular}{@{}l@{}}
     $\mbox{L}_1$\\7, 6 \end{tabular}}
\llhook{-2.9,2.25}{\begin{tabular}{@{}l@{}}
     $\mbox{L}_3$\\7, 6 \end{tabular}}
\llhook{-1.9,2.25}{\begin{tabular}{@{}l@{}}
     $\mbox{L}_5$\\7, 6 \end{tabular}}
\llhook{-0.9,2.25}{\begin{tabular}{@{}l@{}}
     $\mbox{L}_7$\\7, 6 \end{tabular}}
\multiput(1,2)(1,0){4}{\blob}
\rrhook{1.25,2.25}{\begin{tabular}{@{}l@{}}
     $\mbox{L}_2$\\7, 6 \end{tabular}}
\rrhook{2.25,2.25}{\begin{tabular}{@{}l@{}}
     $\mbox{L}_4$\\7, 6 \end{tabular}}
\rrhook{3.25,2.25}{\begin{tabular}{@{}l@{}}
     $\mbox{L}_6$\\7, 6 \end{tabular}}
\rrhook{4.25,2.25}{\begin{tabular}{@{}l@{}}
     $\mbox{L}_8$\\7, 6 \end{tabular}}
\put(0,0){\blob}
\bhook{0,-0.3}{\begin{tabular}{@{}c@{}}
     Lines\\49, 0 \end{tabular}}
\put(0,0){\line(-2,1){4}}
\put(0,0){\line(-3,2){3}}
\put(0,0){\line(-1,1){2}}
\put(0,0){\line(-1,2){1}}
\put(0,0){\line(2,1){4}}
\put(0,0){\line(3,2){3}}
\put(0,0){\line(1,1){2}}
\put(0,0){\line(1,2){1}}
\put(0,4){\line(-2,-1){4}}
\put(0,4){\line(-3,-2){3}}
\put(0,4){\line(-1,-1){2}}
\put(0,4){\line(-1,-2){1}}
\put(0,4){\line(2,-1){4}}
\put(0,4){\line(3,-2){3}}
\put(0,4){\line(1,-1){2}}
\put(0,4){\line(1,-2){1}}
\end{picture}
\end{minipage}
\vspace{\baselineskip}
\caption{Hasse diagrams for analyses, plots and lines from 
the two-phase wheat experiment}
%Example~\ref{eg:WheatStruct}}
\label{f:HasseWheat}
\end{center}
\end{figure}

\begin{table}[htbp]
%\renewcommand{\arraystretch}{1.9}
\begin{center}
\begin{tabular}{lrrrc}
Source & \multicolumn {3}{c}{DF} & Efficiency factor \\
\hline
Intervals & 3 \\
\quad Blocks & & 3          \\
$\mbox{Runs} \nesting{\mbox{Intervals}}$ & 24\\
\quad $\mbox{Plots} \nesting{\mbox{Blocks}}$ & & 24          \\
\qquad Lines   & & & 24 & 1/4  \\
$\mbox{Times} \nesting{\mbox{Intervals}}$ & 24\\
\quad $\mbox{Plots} \nesting{\mbox{Blocks}}$ & & 24          \\
\qquad Lines   & & & 24 & 1/4  \\
$\mbox{Runs} \inter \mbox{Times} \nesting{\mbox{Intervals}}$ & 144 \\
\quad $\mbox{Plots} \nesting{\mbox{Blocks}}$ & & 144          \\
\qquad Lines   & & & 48 & 3/4  \\
\qquad Residual  & & & 96        \\
\hline
Total            & 195
\end{tabular}
\end{center}
\caption{Analysis variance table for the two-phase wheat experiment}
\label{tab:ANOVAWheat}
\end{table}
