\begin{flushleft}\Large\bf A Plant Experiment
\end{flushleft}

A plant experiment is to be conducted to investigate five varieties and two
spray regimes.  Twelve pots containing a single seedling of each variety
will be used and these are to be randomly assigned to six benches on each
of which there are 10 positions so that there are two seedlings of each
variety on each bench.  The two spray regimes are to be randomly assigned
to the benches so that each is applied to the pots on three benches.  The
height gain after six months is measured for each seedling in a pot.

The sets for this experiment are positions, seedlings and regimes and the
tiers are 
${\cal F}_{\rm positions} = \setof{\mbox{Benches}, \mbox{Positions}}$,
${\cal F}_{\rm seedlings} = \setof{\mbox{Varieties}, \mbox{Seedlings}}$ 
and ${\cal F}_{\rm regimes} = \setof{\mbox{Regimes}}$.  
This single-stage experiment involves two randomizations: the randomization
of seedlings to positions and the randomization of regimes to positions.
The randomizations are coincident as the different seedlings and the
different spray regimes are both randomized to the benches.

\begin{figure}[htbp]
\centering
\begin{picture}(10,5)
\put(-1.3,2.9){\begin{tierbox}5 & Varieties\\&\\12 & Seedlings
in V\end{tierbox}}
\put(2.4,3.65){\line(5,-1){4.2}}
\put(2.5,2.5){\line(1,0){0.8}}
\put(3.4,2.5){\line(3,1){1}}
\put(3.4,2.5){\line(3,-1){1}}
\put(3.4,2.5){\blob}
\put(4.6,2.8){\makebox(1,0)[l]{$2 \hspace{\nlevnamesep} S_2$}}
\put(4.6,2.2){\makebox(1,0)[l]{$6 \hspace{\nlevnamesep} S_1$}}
\put(6.5,2.8){\blob}
\put(5.5,2.8){\vector(1,0){1.85}}
\put(7,2.1){\begin{tierbox}10 & Positions in B\\&\\6& Benches\end{tierbox}}
\put(5.5,2.2){\vector(4,-1){2}}
\put(2.5,0.6){\begin{tierbox}2 & Regimes\end{tierbox}}
\put(4.9,0.7){\vector(3,1){2.65}}
\end{picture}
\caption{Coincident randomizations in for the plant experiment}
\label{fig:plant}
\end{figure}

This randomization is summarized in Figure~\ref{fig:plant}, which 
introduces three new features.  The first is that we have a factor from one 
tier, Seedlings (in Varieties), randomized two factors from the second tier, 
Benches and Positions (in Benches). 
To achieve this the seedlings within a variety are split into 6 sets of
two and the sets are randomized to benches while the seedlings within a set 
(within a variety) are randomized to positions within benches.
It is convenient to represent the formation of sets of seedlings by using
pseudofactors (\nocite{Monod92}Monod and Bailey, 1992) $S_1$ and $S_2$ 
for Seedlings, with $6$ and $2$ levels
respectively, so that $\mbox{Seedlings} = S_1\wedge S_2$. We use the
convention that pseudofactors have the same letter as the (real)
factor and are distinguished by subscripts. The lines and black circle between 
Seedlings and the pseudofactors in the diagram portray this splitting of 
Seedlings into the two pseudofactors. 
Now $\mbox{Varieties}
\wedge S_2$ is randomized to Positions in Benches while $S_1$~is
randomized to Benches. Note that the twelve levels of Seedlings, 
and therefore the levels of
both $S_1$ and~$S_2$, must be randomly allocated within each level of
Varieties. If, for example, the levels of~$S_1$ correspond to
different heights of seedling then $S_1$~is crossed with Varieties
rather than nested in Varieties.

Another way of viewing the randomization of Seedlings (in Varieties) 
is that all $60$~levels of the generalized factor 
$\mbox{Varieties} \wedge \mbox{Seedlings}$ is randomized to 
the generalized factor $\mbox{Benches}\wedge \mbox{Positions}$ subject 
to the constraint that Varieties are randomized to Positions within Benches. 

The second new feature is that there are two arrows coming from the 
seedlings tier to the positions tier. All arrows from one tier to
another represent a single randomization.

Finally, coincident randomizations are indicated by two arrowheads at the
same factor. Not every combination of Seedlings and Regimes can occur:
the confounding between Regimes and part of~$S_1$ is indicated by the
fact that both are randomized to Benches.

The structure formulae for this experiment are
\[\begin{array}{c}
  \mbox{6 Benches} \nest \mbox{10 Positions} \\
  (\mbox{5 Varieties} \nest \mbox{12 Seedlings})//(\mbox{6 S}_1)  \\
  \mbox{2 Regimes} \cross \mbox{Varieties}.
\end{array}\]
In this case Varieties has been included in both the second and third 
structure formulae. 
It is inlcuded in the second because Varieties and 
Seedlings form a tier as they were randomized together. 
It is included in the third because the experimenter 
would be interested in the interaction of Regimes with Varieties.

The Hasse diagrams displaying the structures for this experiment are given 
in Figure~\ref{f:HassePlant}. As far as the conditions required for coincident 
randomizations are concerned, this experiment clearly meets the first condition, after the addition of a pseudofactor for Seedlings. Indeed, 
all terms in the structures corresponding to both seedlings and regimes 
are orthogonal to those for positions and the elements of the corresponding 
matrices of efficiencies are either zero or one. 
The second condition is also met with the subspaces corresponding to 
Regimes, Varieties and $\mbox{Regimes} \inter \mbox{Varieties}$ each being 
a subspace of one of those correspnding to Varieties, $\mbox{S}_1$ and 
$\mbox{Seedlings} \nesting{\mbox{Varieties}}$.
The analysis of variance table, that summarizes the relationships 
between subspaces for terms from different structure formulae, 
is given in Table~\ref{tab:ANOVAPlant}.

\begin{figure}[htbp]
\begin{minipage}{0.45\columnwidth}
\centering
\begin{picture}(2,4)(-1,-1)
\put(0,2){\blob}
\llhook{-0.3,2.25}{Universe}
\rrhook{0.3,2.25}{$1$, $1$}
\put(0,1){\blob}
\llhook{-0.3,1.2}{Benches}
\rrhook{0.3,1.2}{6, 5}
\put(0,0){\blob}
\llhook{-0.3,0.2}{$\mbox{Positions}\nesting{\mbox{B}}$}
\rrhook{0.3,0.25}{60, 54}
\put(0,0){\line(0,1){2}}
\end{picture}
\end{minipage}
\hfill
\begin{minipage}{0.45\columnwidth}
\centering
\begin{picture}(4,4)(-2,-1)
\put(0,2){\blob}
\thook{0,2.3}{Universe $1$, $1$}
\put(-1,1){\blob}
\llhook{-1.3,1.25}{\begin{tabular}{@{}c@{}}
     Varieties\\5, 4 \end{tabular}}
\put(1,1){\blob}
\rrhook{1.3,1.25}{\begin{tabular}{@{}c@{}}
     $\mbox{S}_1$\\6, 5 \end{tabular}}
\put(0,0){\blob}
\bhook{0,-0.3}{\begin{tabular}{@{}c@{}}
     $\mbox{Seedlings} \nesting{\mbox{V}}$\\
     60, 50 \end{tabular}}
\multiput(0,0)(-1,1){2}{\line(1,1){1}}
\multiput(0,0)(1,1){2}{\line(-1,1){1}}
\end{picture}
\end{minipage}

\begin{center}
\begin{minipage}{\columnwidth}
\centering
\begin{picture}(4,4)(-2,-1)
\put(0,2){\blob}
\thook{0,2.3}{Universe $1$, $1$}
\put(-1,1){\blob}
\llhook{-1.3,1.25}{\begin{tabular}{@{}c@{}}
     Regimes\\2, 1 \end{tabular}}
\put(1,1){\blob}
\rrhook{1.3,1.25}{\begin{tabular}{@{}c@{}}
     Varieties\\5, 4 \end{tabular}}
\put(0,0){\blob}
\bhook{0,-0.3}{\begin{tabular}{@{}c@{}}
     $\mbox{R} \inter \mbox{V}$\\
     10, 4 \end{tabular}}
\multiput(0,0)(-1,1){2}{\line(1,1){1}}
\multiput(0,0)(1,1){2}{\line(-1,1){1}}
\end{picture}
\end{minipage}
\end{center}
%\vspace{\baselineskip}
\caption{Hasse diagrams for the plant experiment}
\label{f:HassePlant}
\end{figure}

\begin{table}[htbp]
%\renewcommand{\arraystretch}{1.9}
\begin{center}
\begin{tabular}{lrrr}
Source & \multicolumn {3}{c}{DF} \\
\hline
Benches & 5 \\
\quad $\mbox{Seedlings} \nesting{\mbox{Varieties}}$ & & 5          \\
\qquad Regimes   & & & 1  \\
\qquad Residual  & & & 4        \\
$\mbox{Positions}\nesting{\mbox{Benches}}$ & 54\\
\quad Varieties  & & 4  \\
\quad $\mbox{Seedlings} \nesting{\mbox{Varieties}}$ & & 50          \\
\qquad $\mbox{Regimes} \inter \mbox{Varieties}$   & & & 4  \\
\qquad Residual  & & & 46       \\
\hline
Total            & 59
\end{tabular}
\end{center}
\caption{Analysis variance table for the plant experiment}
\label{tab:ANOVAPlant}
\end{table}
