\begin{flushleft}\Large\bf A Two-Phase Potato Storage Experiment
\end{flushleft}

(Prof. Roger Payne, IACR-Rothamsted, kindly provided this experiment.)
This two-phase experiment consists of field and storage phases. In the
field phase four cultivars of potatoes and three fungicides are
investigated using a randomized complete block design with three blocks
each with 12 plots. At the storage phase, the produce from each of the
36 plots is divided into four samples for storage on four pallets and
the produce on a pallet is to be stored for one of four different
lengths of time.  Altogether, there are  12 pallets on each of 12 benches.
The task is to randomize the 144 samples so that the three blocks by four
cultivars are randomized to the 12 benches and the three fungicides with
their four samples for that bench are randomized to the 12 pallets within
each bench.
Finally the four times of storage are randomized to the pallets within
each bench-fungicide combination.
Formally, we identify sets of plots within each block that have the same
cultivar and set up a pseudofactor that indexes these sets.
Similarly we create a pseudofactor that indexes the sets of plots in each
block that received the different fungicides. It turns out that the two
pseudofactors have the same levels for each analysis as Cultivars and
Fungicides, respectively. However, most importantly, the pseudofactors
differ from the factors in being nested within Blocks. The pseudofactors
are used in achieving the desired randomization of the 144 samples to the
144 benches as descried below.

Like the two-phase experiment wheat variery trial, this experiment
involves r-inclusive randomizations. In this case, r-inclusive
randomizations need to be employed because plots within blocks:
(i) had treatments randomized to it in the first randomization; and,
(ii) is to be randomized to benches and to positions within benches
in the second randomization. This experiment also has in common with
the two-phase corn gernination experiment that treatments are
introduced in both the first and second phases.  The sets for this experiment
are pallets, samples, treatments and times and the tiers are
${\cal F}_{\rm pallets} = \setof{\mbox{Benches}, \mbox{Pallets}}$,
${\cal F}_{\rm samples} = \setof{\mbox{Blocks}, \mbox{Plots},
\mbox{Samples}}$,
${\cal F}_{\rm treatments} = \setof{\mbox{Cultivars}, \mbox{Fungicides}}$
and ${\cal F}_{\rm times} = \setof{\mbox{Times}}$.
There are three randomizations: treatments to samples, samples to
pallets and times to pallets. They are illustrated in the randomization
diagram for this experiment given in Figure~\ref{fig:storage}.
The first and second randomizations are r-inclusive and
the second and third are u-inclusive. The randomization of samples
to pallets involves both treatments and samples factors and so the randomized
factors for the second randomization involve both tiers from the first
randomization. The randomization of the times is to the
pallets within each bench-fungicide combination, so the unrandomized
factors for the third randomization come from ${\cal F}_{\rm
treatments}$ and ${\cal F}_{\rm pallets}$.

Note the two pseudofactors, $P_C$ and $P_F$, in Figure~\ref{fig:storage}.
They are nested within Blocks and index the partitions of the Plots with
Blocks that correspond to Cultivars and Fungicides, respectively.
In randomizing the samples to pallets, the levels of Benches and Pallets
are listed in standard order and the levels of the rest of the factors
in the experiment are listed in standard order
according to the levels of Blocks, Samples and the pseudofactors.

\begin{figure}[htbp]
\centering
\begin{picture}(14,6)
\curvedashes[4pt]{0,1,1}
%inner
\closecurve(1,1.3, 8.75,1.3, 9.75,2.3, 9.75,4, 8.75,5, 1,5, 0,4, 0,2.3)
%outer
\closecurve(1,1, 12.8,1, 13.8,2, 13.8,4, 12.8,5.25, 1,5.25, 0,4, 0,2)
\put(0.1,3){\begin{tierbox}4 & Cultivars \\ \\ \\ \\ \\
                           3 & Fungicides\end{tierbox}}
\put(3.55,3.1){\blob}
\put(3.5,3.1){\line(-1,2){0.6}}
\put(3.5,3.1){\line(-1,-2){0.6}}
\put(3.55,3.1){\vector(1,0){0.65}}
%top lines
\put(9.7,4.4){\line(-1,0){6.85}}
\put(9.675,4.25){$\Diamond$}
\put(9.975,4.4){\vector(1,0){0.5}}
\put(9.7,4.4){\line(-2,-5){0.2}}
\put(9.535,4.025){\blob}
\put(9.535,4.025){\line(-4,-1){1.7}}
\put(9.535,4.025){\line(-1,-4){0.1}}
%bottom lines
\put(9.7,1.85){\line(-1,0){6.8}}
\put(9.675,1.7){$\Diamond$}
\put(9.975,1.85){\vector(1,0){0.5}}
\put(9.7,1.85){\line(-2,5){0.2}}
\put(9.535,2.22){\blob}
\put(9.535,2.22){\line(-4,1){1.7}}
\put(9.535,2.22){\line(-1,4){0.1}}
%Block-Plots randomization
\put(3.75,3){\begin{tierbox}3 & Blocks\\12 & Plots in B\\
4 & Samples in B$\wedge$P\end{tierbox}}
\put(8.3,3.1){\line(3,2){0.4}}
\put(8.3,3.1){\line(3,-2){0.4}}
\put(8.3,3.1){\blob}
\put(7.8,3.1){\line(1,0){0.5}}
\put(8.8,3.35){\makebox(1,0)[l]{$4 \hspace{\nlevnamesep} P_C$}}
\put(8.8,2.85){\makebox(1,0)[l]{$3 \hspace{\nlevnamesep} P_F$}}
\put(10,3){\begin{tierbox}12 & Benches\\ \\ \\ \\ \\ 12 & Pallets in B
\end{tierbox}}
%Times randomization
\put(3.7,0.245){\vector(1,0){0.82}}
\put(4.5,0.1){$\Diamond$}
\put(4.8,0.245){\line(1,2){1.05}}
\put(4.8,0.245){\line(4,1){5.75}}
\put(1.6,0.1){\begin{tierbox}4 & Times\end{tierbox}}
\end{picture}
\caption{Unrandomized- and randomized-inclusive randomizations 
in the potato storage experiment}
\label{fig:storage}
\end{figure}








