\documentclass[a4paper,10pt]{article}
%next two lines removed in May 2007
%\documentclass[a4paper,12pt]{article}
%\usepackage{rss}
% using amsmath means the amsTeX commands in RABStyle.sty
% need to be commented out
\usepackage{amsmath}
\usepackage{fancybox}
\usepackage{multibox}
\usepackage{booktabs}
\usepackage{theorem}
%\usepackage{fancyhdr}
\usepackage{curves}
\usepackage{latexsym}
\usepackage{amssymb}
\usepackage{natbib}
\usepackage{rotating}
\title{Multitiered experiments: II. Structure and analysis}

%setting for margins
\setlength{\oddsidemargin}{0cm}
\setlength{\evensidemargin}{0cm}
\setlength{\marginparwidth}{0cm}
\setlength{\marginparsep}{0cm}
\setlength{\textwidth}{15.9cm}
\setlength{\topmargin}{-1.3cm}
\setlength{\headheight}{0.75cm}
\setlength{\headsep}{0.50cm}
\setlength{\textheight}{24.8cm}
\setlength{\hoffset}{0cm}
\setlength{\hoffset}{0cm}
\setlength{\footskip}{1.1cm}


%rule widths in tables
\heavyrulewidth=\lightrulewidth


%commands to define new mathematical operators, relations, binary operators,
%etc.
\DeclareMathOperator{\mean}{mean}
\DeclareMathOperator{\RMSE}{RMSE}
\DeclareMathOperator{\pow}{pow}
\DeclareMathOperator{\gf}{gf}
\DeclareMathOperator{\op}{op}
\DeclareMathOperator{\ar1}{ar1}
\DeclareMathOperator{\td}{td}
\DeclareMathOperator{\uc}{uc}
\DeclareMathOperator{\us}{us}
\DeclareMathOperator{\ush}{ush}
\DeclareMathOperator{\spl}{spl}
\DeclareMathOperator{\dev}{dev}
\DeclareMathOperator{\lin}{lin}
\newcommand{\newmathop}[2]{\def#1{\mathop{#2}\nolimits}}
\newcommand{\newmathopwithlimits}[2]{\def#1{\mathop{#2}}}
\newcommand{\newmathbin}[2]{\def#1{\mathbin{#2}}}
\newcommand{\newmathrel}[2]{\def#1{\mathrel{#2}}}
%\newcommand{\E}[1]{{\mathop{\mtxt{E}\kern-1pt}\nolimits}\left(#1\right)}
\newcommand{\Cov}[1]{{\mathop{\mtxt{cov}\kern-1pt}\nolimits}\left(#1\right)}
\newcommand{\Var}[1]{{\mathop{\mtxt{var}\kern-1pt}\nolimits}\left(#1\right)}
%\newcommand{\fixcont}[1]{\ensuremath{q({#1})}}
\newcommand{\fixcont}[1]{\ensuremath{\theta_{#1}}}
\newcommand{\trace}{\mathop{\mtxt{trace}}}
\newcommand{\rank}{\mathop{\mtxt{rank}}}
\newcommand{\image}{\mathop{\mtxt{Im}}\nolimits}
\newcommand{\setof}[1]{\ensuremath{\left\{#1\right\}}}
\newcommand{\setofall}[2]{\setof{#1:#2}}
\newcommand{\card}[1]{\left|#1\right|}
\newcommand{\conf}{\mathbin{\leftarrow}}
\newcommand{\rterm}{*}
\newcommand{\combine}{\mathbin{\vartriangleright}}
\newcommand{\scombine}{\mathbin{\vartriangleright}}
\newcommand{\resid}{\mathbin{\vdash}}
\newcommand{\sresid}{\mathbin{\vdash}}
\newmathbin{\joint}{\square}
\newcommand{\meet}{{\mathbin{\wedge}}}
\newcommand{\inter}{\mathbin{\#}}
\newcommand{\nesting}[1]{\left[#1\right]}
\newcommand{\curlys}{\Gamma}
\newcommand{\curlyt}{\Upsilon}
\newcommand{\summ}{\mathop{\sum\nolimits'}}
\newcommand{\ems}{\mathop{\mtxt{EMS}}\nolimits}
\newcommand{\ms}{\mathop{\mtxt{MS}}\nolimits}
%\text format as text and can include accents whereas \mtxt formats as a symbol
%- I am using this \mathsrm for factor names; with \text equivalent to \text when amsmath loaded
\newcommand{\mtxt}{\mathrm}

% for nesting and crossing operators
\newcommand{\nest}{\mathbin{/}}
\newcommand{\pseudo}{\mathbin{//}}
\newcommand{\cross}{\mathbin{*}}
\newcommand{\add}{\mathbin{+}}
\newcommand{\drop}{\mathbin{-}}
\newcommand{\modsep}{\mathbin{\mid}}



\newtheorem{lemma}{Lemma}
\newtheorem{theorem}{Theorem}
\newtheorem{thm}[theorem]{Theorem}
\newtheorem{corollary}[theorem]{Corollary}
\newtheorem{rules}{Rule}

%\plaintheorems
\theoremstyle{plain} \theorembodyfont{\rmfamily}
\newtheorem{definition}{Definition}
\newtheorem{defn}[definition]{Definition}
\newtheorem{egg}{Example}
\newenvironment{cont}[1]{\begin{trivlist}
\item \textbf{Example \ref{#1} (continued)}}{\end{trivlist}}

% enumerate-type environment with lower-case roman numerals
\newenvironment{enumeroman}{\renewcommand{\theenumi}{\roman{enumi}}
\renewcommand{\labelenumi}{(\theenumi)}\begin{enumerate}}
{\end{enumerate}}

%\newenvironment{proof}{\trivlist
%%
%% Define proof environment so that proofs just go between
%% \begin{proof} and \end{proof}.
%% The proof environment puts a box at the end of a proof
%%
%\item[\hskip \labelsep {\sc Proof:}\enskip]}%
%{\unskip\nobreak\hskip 2em plus 1fil\nobreak%
%\fbox{\rule{0ex}{1ex}\hspace{1ex}\rule{0ex}{1ex}}%
%\parfillskip=0pt \endtrivlist}
%
% TABULARRELATE & INVISTABULARRELATE
% - environments for producing a tabular and invisible tabular containing factor
%   relationship formulae
%
\newenvironment{tabularrelate}{\begin{tabular}{@{\protect\strutt}|c|l|} \hline
       Formula &  \multicolumn{1}{c|}{Formula} \\  \hline}%
{\hline\end{tabular}}
\newenvironment{invistabularexpstr}{\begin{tabular}{@{\protect\strutt}cl}
       Formula &  \multicolumn{1}{c}{Formula} \\}%
{\end{tabular}}
%
% DISPLAYRELATE & INVISDISAPLAYRELATE
% - environments for displaying and invisible displaying of factor relationship formulae
%
\newenvironment{displayrelate}{\begin{center}\begin{tabularrelate}}%
{\end{tabularrelate}\end{center}}
\newenvironment{invisdisplayrelate}{\begin{center}\begin{invistabularrelate}}%
{\end{invistabularrelate}\end{center}}

%commands for pictures
\setlength{\unitlength}{1cm}
\newcommand{\blob}{\circle*{0.2}}
\newcommand{\nonorthbox}{\put(0,0){$\Box$}}
\newcommand{\nonorthcircle}{\put(0,0){\circle{0.2}}}
\newcommand{\orthcircle}{\put(0,0){\circle{0.2}}\put(-0.09,-0.05){{\tiny$\bot$}}}
\newcommand{\restrict}{\put(0,0){\circle{0.2}}\put(0,0){\circle*{0.075}}}
\newcommand{\dottydiamond}{\put(0,0){$\lozenge$}\put(0.035,0.055){{\tiny$\blacklozenge$}}}
\newcommand{\makepseudo}{\put(0,0){$\blacklozenge$}}
\newcommand{\llhook}[2]{\put(#1){\makebox(0,0)[r]{#2}}}
\newcommand{\rrhook}[2]{\put(#1){\makebox(0,0)[l]{#2}}}
\newcommand{\thook}[2]{\put(#1){\makebox(0,0)[b]{#2}}}
\newcommand{\bhook}[2]{\put(#1){\makebox(0,0)[t]{#2}}}

%for pictures of tiers
\newlength{\nlevnamesep}
\setlength{\nlevnamesep}{0.4em}
\newsavebox{\savetier}
\newenvironment{tierbox}{\begin{lrbox}{\savetier}
\begin{tabular}{r@{\hspace{\nlevnamesep}}l}}{\end{tabular}\end{lrbox}
\ovalbox{\usebox{\savetier}}}

%for efficiency matrices
\newlength{\perch}
\newcommand{\fish}[2]{\settowidth{\perch}{\mbox{#1}}\makebox[\perch]{$#2$}}

\hyphenation{inter-tier pseudo-factor pseudo-factors}

\begin{document}
\begin{flushleft}\Large\bf Duplicated wheat measurements
\end{flushleft}

\label{eg:Wheat2} Example~9 of \cite{BrBa:mult} is an experiment that
consists of a field phase and a laboratory phase. In the field phase 49
lines of wheat are investigated using a randomized complete-block
design with four blocks. Here the laboratory phase is modified by
supposing that the procedure described by \cite{BrBa:mult} is repeated
on a second occasion. That is, two samples will be obtained from each
plot and one of them processed on the first occasion and the other on
the second occasion. Figure~\ref{fig:wheat2} gives the randomization
diagram for the modified experiment. Recall that a $7 \times 7$
balanced lattice square design with four replicates is used to assign
the blocks, plots and lines to four intervals in each occasion. In each
interval on one occasion there are seven runs at which samples are
processed at seven consecutive times. Pseudofactors are introduced for
lines and plots in order to define the design of the second phase. The
sets of objects for this experiment are analyses, samples and lines.

\begin{figure}[htbp]
\setlength{\unitlength}{1cm} \centering \small
\begin{picture}(12.7,3.5)(0,0)
\curvedashes[4pt]{0,1,1} \closecurve(0.1,2.0, 0.2,1.3, 7.5,1.2, 7.6,2.0
,7.5,2.8, 0.2,2.7)
\put(0.1,1.9){\begin{tierbox}49 & Lines\hspace{250pt}\end{tierbox}} \put(2.65,2.0){\vector(1,0){1.2}}
\put(2,2.0){\line(1,0){0.5}}
\put(2.4,2.0){\blob}
\put(2.4,2.0){\line(2,3){0.8}}
\put(2.4,2.0){\line(2,-3){0.8}}
\put(3.4,3.2){\makebox(1,0)[l]{$7 \hspace{\nlevnamesep}
                                {\rm L}_1,{\rm L}_3,{\rm L}_5,{\rm L}_7$}}
\put(3.4,0.8){\makebox(1,0)[l]{$7 \hspace{\nlevnamesep}
                                {\rm L}_2,{\rm L}_4,{\rm L}_6,{\rm L}_8$}}
\put(3.5,1.9){\begin{tierbox}4 & Blocks\\49 & Plots in B\\
                             2 & Samples in B, P\end{tierbox}}
\put(5.7,2.375){\vector(1,0){4.1}}
\put(7.0,2.0){\line(-1,0){1.0}}
\put(7.0,2.0){\blob}
\put(7.0,2.0){\line(2,3){0.805}}
\put(7.0,2.0){\line(2,-3){0.85}}
\put(8.0,3.15){\makebox(1,0)[l]{$7\hspace{\nlevnamesep} {\rm P}_1$}}
\put(8.0,0.75){\makebox(1,0)[l]{$7\hspace{\nlevnamesep} {\rm P}_2$}}
\put(6.7,3.10){\makepseudo}
\put(6.7,0.65){\makepseudo}
\put(7.81,3.2){\line(-1,0){2.2}}
\put(7.84,0.75){\line(-1,0){2.2}}
\put(8.8,3.2){\vector(1,0){1.05}}
\put(8.8,0.75){\vector(1,0){1.05}}
\put(8.0,1.6){\makebox(1,0)[l]{$2\hspace{\nlevnamesep} {\rm S}_1$}}
\put(7.81,1.6){\line(-1,0){1.0}}
\put(8.8,1.6){\vector(1,0){1.05}}
\put(9.25,1.875){\begin{tierbox}\quad 7 & Runs in O, I\\&\\
                                 4 & Intervals in O\\&\\
                                 2 & Occasions\\&\\ 7 & Times in O, I\end{tierbox}}
\put(1.1,0.1){\makebox(0,0){49 lines}}
\put(5.4,0.1){\makebox(0,0){392 samples}}
\put(11.1,0.1){\makebox(0,0){392 analyses}}
\end{picture}
\caption{Randomized-inclusive randomizations in the wheat experiment} \label{fig:wheat2}
\end{figure}

The covariance matrix under the randomizations is
\begin{eqnarray}
\mathbf{C} & = & \xi_0 \mathbf{P}_0 + \xi_\mtxt{O}
\mathbf{P}_\mtxt{O} + \xi_\mtxt{OI} \mathbf{P}_\mtxt{OI} +
\xi_\mtxt{OIR} \mathbf{P}_\mtxt{OIR} + \xi_\mtxt{OIT}
\mathbf{P}_\mtxt{OIT} +
\xi_\mtxt{OITR} \mathbf{P}_\mtxt{OITR} \nonumber\\
& & \qquad + 2\eta_0 \mathbf{Q}_0 + 2\eta_\mtxt{B}
\mathbf{Q}_\mtxt{B} + 2\eta_\mtxt{BP} \mathbf{Q}_\mtxt{BP} +
2\eta_\mtxt{BPS} \mathbf{Q}_\mtxt{BPS}. \nonumber
\end{eqnarray}
Randomized-inclusive randomizations were used in this experiment, as
the outcome of the randomization of lines to samples had to be known
before the samples could be randomized to analyses.
The plots pseudofactors~$\mtxt{P}_1$ and~$\mtxt{P}_2$ were
used to ensure appropriate partial confounding of sources from the
treatments tier with sources in the analyses tier. As in
Example~\ref{eg:cotton}, there are no idempotents for the Plot
pseudofactors in $\mathbf{C}$ because the pseudo\-factors are irrelevant
to the first randomization, that of treatments to samples, and are not
one of the unrandomized factors, that gives rise to covariance, in the
other randomization. However, as in Example~\ref{eg:cotton},
$\mathbf{Q}_\mtxt{BP}$ can be rewritten as the sum of three
$\mathbf{Q}^*$-matrices each with coefficient $\eta_{\mtxt{BP}}$.
This results in the coefficient $\eta_{\mtxt{BP}}$ occuring with
three different $\xi$-coefficients in the skeleton analysis of variance
in Table~\ref{tab:ANOVAWheat2}, which is an extended version of the
decomposition table given by \cite{BrBa:decomp} in their Example 4.
To obtain structure balance, the projection space of the idempotent $\mathbf{Q}_\mtxt{BPS}$ is decomposed as the sum of
five subspaces involving the pseudofactor $\mtxt{S}_1$ so that
$\mathbf{Q}_\mtxt{BPS}$ can be rewritten as the sum of five
$\mathbf{Q}^*$-matrices each with coefficient $\eta_{\mtxt{BPS}}$. As
a consequence, the coefficient $\eta_{\mtxt{BPS}}$ occurs with five
different $\xi$-coefficients in Table~\ref{tab:ANOVAWheat2}.

\begin{center}
\begin{tabular}{lr}
\toprule
\multicolumn{2}{c}{analyses tier} \\
\midrule
Source & df    \\
\midrule
Mean & 1    \\
    Occasions & 1    \\
    $\mtxt{Intervals}\nesting{\mtxt{O}}$ & 6   \\
$\mtxt{Runs} \nesting{\mtxt{O}\wedge\mtxt{I}}$ & 48  \\
$\mtxt{Times}\nesting{\mtxt{O}\wedge\mtxt{I}}$ & 48  \\
$\mtxt{R}\inter\mtxt{T}\nesting{\mtxt{O}\wedge\mtxt{I}}$ & 288  \\
\bottomrule
\end{tabular}
\end{center}

\begin{center}
\begin{tabular}{lrclr}
\toprule
\multicolumn{2}{c}{analyses tier} & & \multicolumn{2}{c}{plots tier} \\
\cmidrule{1-2} \cmidrule{4-5}
Source & df & & Source & df   \\
\midrule
Mean & 1 & & Mean & 1   \\
\midrule
    Occasions & 1 & & $\mtxt{S}_1$ & 1   \\
\midrule
    $\mtxt{Intervals}\nesting{\mtxt{O}}$ & 6 & & Blocks & 3  \\
 & & &  $\mtxt{S}_1\inter\mtxt{B}$ & 3   \\
\midrule
 $\mtxt{Runs} \nesting{\mtxt{O}\wedge\mtxt{I}}$ & 48 & &
       $\mtxt{P}_1 \nesting{\mtxt{B}}$ &  24  \\
 & & &  $\mtxt{S}_1\inter\mtxt{P}_1\nesting{\mtxt{B}}$ & 24   \\
\midrule
$\mtxt{Times}\nesting{\mtxt{O}\wedge\mtxt{I}}$ & 48 & &
       $\mtxt{P}_2 \nesting{\mtxt{B}}$ &  24 \\
 & & &  $\mtxt{S}_1\inter\mtxt{P}_2\nesting{\mtxt{B}}$ & 24  \\
\midrule
$\mtxt{R}\inter\mtxt{T}\nesting{\mtxt{O}\wedge\mtxt{I}}$ & 288 & &
       $\mtxt{Plots}\nesting{\mtxt{B}}_{\sresid}$ &  144 \\

& & &
$\mtxt{Samples}\nesting{\mtxt{B}\wedge\mtxt{P}}_{\sresid}$
              & 144  \\
\bottomrule
\end{tabular}
\end{center}

\begin{center}
\begin{tabular}{lrclrcclr}
\toprule
\multicolumn{2}{c}{analyses tier} & & \multicolumn{2}{c}{plots tier} & &
\multicolumn{3}{c}{lines tier}  \\
\cmidrule{1-2} \cmidrule{4-5} \cmidrule{7-9}
Source & df & & Source & df & & eff & Source & df  \\
\midrule
Mean & 1 & & Mean & 1 & & & Mean & 1  \\
\midrule
    Occasions & 1 & & $\mtxt{S}_1$ & 1 & & & &  \\
\midrule
    $\mtxt{Intervals}\nesting{\mtxt{O}}$ & 6 & & Blocks & 3 & & & &  \\
 & & &  $\mtxt{S}_1\inter\mtxt{B}$ & 3 & & &  \\
\midrule 
$\mtxt{Runs} \nesting{\mtxt{O}\wedge\mtxt{I}}$ & 48 & &
       $\mtxt{P}_1 \nesting{\mtxt{B}}$ &  24 &
        &\raisebox{8.5pt}[9pt][5pt]{}$\frac{1}{4}$ & $\mtxt{Lines}_{\mtxt{R}}$  & 24 \\
& & &  $\mtxt{S}_1\inter\mtxt{P}_1\nesting{\mtxt{B}}$ & 24 & & &  \\
\midrule
$\mtxt{Times}\nesting{\mtxt{O}\wedge\mtxt{I}}$ & 48 & &
       $\mtxt{P}_2 \nesting{\mtxt{B}}$ &  24 &
        &\raisebox{8.5pt}[9pt][5pt]{}$\frac{1}{4}$ & $\mtxt{Lines}_{\mtxt{T}}$ & 24 \\
 & & &  $\mtxt{S}_1\inter\mtxt{P}_2\nesting{\mtxt{B}}$ & 24 & & &  \\
\midrule
$\mtxt{R}\inter\mtxt{T}\nesting{\mtxt{O}\wedge\mtxt{I}}$ & 288 & &
       $\mtxt{Plots}\nesting{\mtxt{B}}_{\sresid}$ &  144 &
        &\raisebox{8.5pt}[9pt][5pt]{}$\frac{3}{4}$ & $\mtxt{Lines}_{\mtxt{R}}$ & 24 \\
& & & & & &\raisebox{8.5pt}[9pt][5pt]{}$\frac{3}{4}$ &
        $\mtxt{Lines}_{\mtxt{T}}$ & 24 \\
& & & & & & & Residual  & 96   \\
\cmidrule{4-9}
& & &
$\mtxt{Samples}\nesting{\mtxt{B}\wedge\mtxt{P}}_{\sresid}$
              & 144 & & & &  \\
\bottomrule
\end{tabular}
\end{center}


\begin{center}
\begin{tabular}{*{2}{lr@{\hspace{0.5em}}c@{\hspace{0.5em}}}c@{\hspace{0.5em}}lr@{\hspace{0.5em}}c@{\hspace{0.5em}}%
*{7}{@{}c@{}}c@{}c}
\toprule
\multicolumn{2}{c}{analyses tier} & & \multicolumn{2}{c}{plots tier} & &
\multicolumn{3}{c}{lines tier} & & \multicolumn{9}{c}{EMS} \\
\cmidrule{1-2} \cmidrule{4-5} \cmidrule{7-9} \cmidrule{11-19}
Source & df & & Source & df & & eff &Source & df & & $\phi_{\mtxt{OIRT}}$ & $\phi_{\mtxt{OIT}}$ & $\phi_{\mtxt{OIR}}$ & $\phi_{\mtxt{OI}}$ & $\phi_{\mtxt{O}}$ & $\phi_{\mtxt{BPS}}$ & $\phi_{\mtxt{BP}}$ & $\phi_{\mtxt{B}}$ & $q(.)$ \\
\midrule
Mean & 1 & & Mean & 1 & & & Mean & 1 & & 1 & 7 & 7 & 49 & 96 & 1 & 2 & 98 & $\phi_\mtxt{a} + \phi_\mtxt{p} + q(\mu)$ \\
\midrule
    Occasions & 1 & & $\mtxt{S}_1$ & 1 &  & & & & &  1 & 7 & 7 & 49 & 96 & 1 &  &  &  \\
\midrule
    $\mtxt{Intervals}\nesting{\mtxt{O}}$ & 6 & & Blocks & 3 & & & & & &
                                                   1 & 7 & 7 & 49 &  & 1 & 2 & 98 & \\
   & & &  $\mtxt{S}_1\inter\mtxt{B}$ & 3 & & & & & &
                                                   1 & 7 & 7 & 49 &  & 1 &  &  &  \\
\midrule
$\mtxt{Runs} \nesting{\mtxt{O}\wedge\mtxt{I}}$ & 48 & &
       $\mtxt{P}_1 \nesting{\mtxt{B}}$ &  24 &
        &\raisebox{8.5pt}[9pt][5pt]{}$\frac{1}{4}$ & $\mtxt{Lines}_{\mtxt{R}}$  & 24 &
        & 1 &  & 7 &  &  & 1 & 2 &  & \raisebox{8.5pt}[9pt][5pt]{}$\frac{1}{4}q(L_R)$ \\
& & &  $\mtxt{S}_1\inter\mtxt{P}_1\nesting{\mtxt{B}}$ & 24 & & & & & &
                                                   1 &  & 7 &  &  & 1 &  &  &  \\
\midrule
$\mtxt{Times}\nesting{\mtxt{O}\wedge\mtxt{I}}$ & 48 & &
       $\mtxt{P}_2 \nesting{\mtxt{B}}$ &  24 &
        &\raisebox{8.5pt}[9pt][5pt]{}$\frac{1}{4}$ & $\mtxt{Lines}_{\mtxt{T}}$ & 24 &
        &  1 & 7 &  &  &  & 1 & 2 &  & \raisebox{8.5pt}[9pt][5pt]{}$ \frac{1}{4}q(L_T)$\\
& & &  $\mtxt{S}_1\inter\mtxt{P}_2\nesting{\mtxt{B}}$ & 24 & & & & & &
                                                   1 & 7 &  &  & & 1 &  &  & \\
\midrule
$\mtxt{R}\inter\mtxt{T}\nesting{\mtxt{O}\wedge\mtxt{I}}$ & 288 & &
       $\mtxt{Plots}\nesting{\mtxt{B}}_{\sresid}$ &  144 &
        &\raisebox{8.5pt}[9pt][5pt]{}$\frac{3}{4}$ & $\mtxt{Lines}_{\mtxt{R}}$ & 24 &
        &  1 &  &  &  &  & 1 & 2 &  & \raisebox{8.5pt}[9pt][5pt]{}$\frac{3}{4}q(L_R)$ \\
& & & & & &\raisebox{8.5pt}[9pt][5pt]{}$\frac{3}{4}$ &
        $\mtxt{Lines}_{\mtxt{T}}$ & 24 & &
          1 &  &  &  &  & 1 & 2 &  & \raisebox{8.5pt}[9pt][5pt]{}$\frac{3}{4} q(L_T)$\\
& & & & & & & Residual  & 96 & & 1 &  &  &  &  & 1 & 2 &  &   \\
\cmidrule{4-19}
& & &
$\mtxt{Samples}\nesting{\mtxt{B}\wedge\mtxt{P}}_{\sresid}$
              & 144 & & & & & & 1 &  &  &  &  & 1 & \ &  &  \\
\bottomrule
\end{tabular}
\end{center}

\begin{sideways}
%\begin{center}
	\begin{tabular}{*{2}{lr@{\hspace{0.5em}}c@{\hspace{0.5em}}}c@{\hspace{0.5em}}lr@{\hspace{0.5em}}c@{\hspace{0.5em}}*{9}{@{}c@{}}c@{}c}
		\toprule
		\multicolumn{2}{c}{analyses tier} & & \multicolumn{2}{c}{plots tier} & & \multicolumn{3}{c}{lines tier} & & \multicolumn{10}{c}{EMS} & \\
		\cmidrule{1-2} \cmidrule{4-5} \cmidrule{7-9} \cmidrule{11-21}
		Source & df & & Source & df & & eff &Source & df & & $\phi_{\mtxt{OIRT}}$ & $\phi_{\mtxt{OIT}}$ & $\phi_{\mtxt{OIR}}$ & $\phi_{\mtxt{OI}}$ & $\phi_{\mtxt{O}}$ & $\phi_{\mtxt{BPS}}$ & $\phi_{\mtxt{BP}}$ & $\phi_{\mtxt{B}}$ & $\phi_\mtxt{a} + \phi_\mtxt{p}$ & $\fixcont{\cdot}$ & \\
		\midrule
		Mean & 1 & & Mean & 1 & & & Mean & 1 & & 1 & 7 & 7 & 49 & 96 & 1 & 2 & 98 & 1 &  $\fixcont{\mu}$ & \\
		\midrule
		Occasions & 1 & & $\mtxt{S}_1$ & 1 &  & & & & &  1 & 7 & 7 & 49 & 96 & 1 &  &  &  & & \\
		\midrule
		$\mtxt{Intervals}\nesting{\mtxt{O}}$ & 6 & & Blocks & 3 & & & & & & 1 & 7 & 7 & 49 &  & 1 & 2 & 98 & & & \\
		& & &  $\mtxt{S}_1\inter\mtxt{B}$ & 3 & & & & & & 1 & 7 & 7 & 49 &  & 1 &  &  & & & \\
		\midrule
		$\mtxt{Runs} \nesting{\mtxt{O}\wedge\mtxt{I}}$ & 48 & & $\mtxt{P}_1 \nesting{\mtxt{B}}$ &  24 &
		&\raisebox{8.5pt}[9pt][5pt]{}$\frac{1}{4}$ & $\mtxt{Lines}_{\mtxt{R}}$  & 24 &
		& 1 &  & 7 &  &  & 1 & 2 &  & & $\fixcont{\mtxt{OIR}^\rterm\conf\mtxt{BP}_1^\rterm\conf\mtxt{L}_\mtxt{R}}$ & \\
		& & &  $\mtxt{S}_1\inter\mtxt{P}_1\nesting{\mtxt{B}}$ & 24 & & & & & & 1 &  & 7 &  &  & 1 &  &  &  & & \\
		\midrule
		$\mtxt{Times}\nesting{\mtxt{O}\wedge\mtxt{I}}$ & 48 & & $\mtxt{P}_2 \nesting{\mtxt{B}}$ &  24 &
		&\raisebox{8.5pt}[9pt][5pt]{}$\frac{1}{4}$ & $\mtxt{Lines}_{\mtxt{T}}$ & 24 &
		&  1 & 7 &  &  &  & 1 & 2 &  & & $\fixcont{\mtxt{OIT}^\rterm\conf\mtxt{BP}_2^\rterm\conf\mtxt{L}_\mtxt{T}}$ & \\
		& & &  $\mtxt{S}_1\inter\mtxt{P}_2\nesting{\mtxt{B}}$ & 24 & & & & & & 1 & 7 &  &  & & 1 &  &  & & & \\
		\midrule
		$\mtxt{R}\inter\mtxt{T}\nesting{\mtxt{O}\wedge\mtxt{I}}$ & 288 & & $\mtxt{Plots}\nesting{\mtxt{B}}_{\sresid}$ &  144 &
		&\raisebox{8.5pt}[9pt][5pt]{}$\frac{3}{4}$ & $\mtxt{Lines}_{\mtxt{R}}$ & 24 &
		&  1 &  &  &  &  & 1 & 2 &  & & $\fixcont{\mtxt{OIRT}^\rterm\conf\mtxt{BP}_{\sresid}^\rterm\conf\mtxt{L}_\mtxt{R}}$ & \\
		& & & & & &\raisebox{8.5pt}[9pt][5pt]{}$\frac{3}{4}$ &
		$\mtxt{Lines}_{\mtxt{T}}$ & 24 & & 1 &  &  &  &  & 1 & 2 &  & & $\fixcont{\mtxt{OIRT}^\rterm\conf\mtxt{BP}_{\sresid}^\rterm\conf\mtxt{L}_\mtxt{T}}$ & \\
		& & & & & & & Residual  & 96 & & 1 &  &  &  &  & 1 & 2 &  & & &  \\
		\cmidrule{4-20}
		& & &
		$\mtxt{Samples}\nesting{\mtxt{B}\wedge\mtxt{P}}_{\sresid}$
		& 144 & & & & & & 1 &  &  &  &  & 1 &  &  &  & & \\
		\bottomrule
	\end{tabular}
%\end{center}
\end{sideways}


\clearpage
\begin{table}
\caption{\label{tab:ANOVAWheat2}Skeleton analysis of variance for
Example~\ref{eg:Wheat2}}
\begin{center}
\begin{tabular}{lr|lr|clr|l}
\multicolumn{2}{c|}{analyses tier} & \multicolumn{2}{c|}{plots tier} &
\multicolumn{3}{c|}{lines tier} & \\
Source & df & Source & df & eff &Source & df & \multicolumn{1}{c}{EMS} \\
\hline
Mean & 1 & Mean & 1 & & Mean & 1 & $\xi_0 + \eta_0$ \\
\hline
    Occasions & 1 & $\mtxt{S}_1$ & 1 &  & & & $\xi_\mtxt{O} + \eta_\mtxt{BPS}$ \\
\hline
    $\mtxt{Intervals}\nesting{\mtxt{O}}$ & 6 & Blocks & 3 &  & & &
                                                  $\xi_\mtxt{OI} + \eta_\mtxt{B}$ \\
\cline{3-8} & &  $\mtxt{S}_1\inter\mtxt{B}$ & 3 & & & &
                                                  $\xi_\mtxt{OI} + \eta_\mtxt{BPS}$ \\
\hline $\mtxt{Runs} \nesting{\mtxt{O}\wedge\mtxt{I}}$ & 48 &
       $\mtxt{P}_1 \nesting{\mtxt{B}}$ &  24
        &\raisebox{8.5pt}[9pt][5pt]{}$\frac{1}{4}$ & $\mtxt{Lines}_{\mtxt{R}}$  & 24
        &\raisebox{8.5pt}[9pt][5pt]{}$\xi_\mtxt{OIR} + \eta_\mtxt{BP} + \frac{1}{4}q(L_R)$ \\
\cline{3-8}
& &  $\mtxt{S}_1\inter\mtxt{P}_1\nesting{\mtxt{B}}$ & 24 & & & &
                                                  $\xi_\mtxt{OIR} + \eta_\mtxt{BPS}$ \\
\hline
$\mtxt{Times}\nesting{\mtxt{O}\wedge\mtxt{I}}$ & 48 &
       $\mtxt{P}_2 \nesting{\mtxt{B}}$ &  24
        &\raisebox{8.5pt}[9pt][5pt]{}$\frac{1}{4}$ & $\mtxt{Lines}_{\mtxt{T}}$ & 24
        &\raisebox{8.5pt}[9pt][5pt]{}$\xi_\mtxt{OIT} + \eta_\mtxt{BP} + \frac{1}{4}q(L_T)$\\
\cline{3-8}
 & &  $\mtxt{S}_1\inter\mtxt{P}_2\nesting{\mtxt{B}}$ & 24 & & & &
                                                  $\xi_\mtxt{OIT} + \eta_\mtxt{BPS}$ \\
\hline
$\mtxt{R}\inter\mtxt{T}\nesting{\mtxt{O}\wedge\mtxt{I}}$ & 288 &
       $\mtxt{Plots}\nesting{\mtxt{B}}_{\sresid}$ &  144
        &\raisebox{8.5pt}[9pt][5pt]{}$\frac{3}{4}$ & $\mtxt{Lines}_{\mtxt{R}}$ & 24
        &\raisebox{8.5pt}[9pt][5pt]{}$\xi_\mtxt{OIRT} + \eta_\mtxt{BP} + \frac{3}{4}q(L_R)$ \\
\cline{5-8}
& & & &\raisebox{8.5pt}[9pt][5pt]{}$\frac{3}{4}$ &
        $\mtxt{Lines}_{\mtxt{T}}$ & 24 &
         \raisebox{8.5pt}[9pt][5pt]{}$\xi_\mtxt{OIRT} + \eta_\mtxt{BP} + \frac{3}{4} q(L_T)$\\
\cline{5-8}
& & & & & Residual  & 96 & $\xi_\mtxt{OIRT} + \eta_\mtxt{BP}$  \\
\cline{3-8} & &
$\mtxt{Samples}\nesting{\mtxt{B}\wedge\mtxt{P}}_{\sresid}$
              & 144 & & & & $\xi_\mtxt{OIRT} + \eta_\mtxt{BPS}$ \\
\hline
\end{tabular}
\end{center}
\end{table}


\end{document}
