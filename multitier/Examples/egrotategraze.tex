\begin{flushleft}\Large\bf An Improperly Replicated Rotational Grazing Experiment
\end{flushleft}

One way to reduce herd size in grazing trials is to use a single
herd to graze the replicates of each treatment in turn.
For example, suppose an experiment is to be conducted to investigate the
effects of three levels of pasture availability on the weight gain of cattle.
Further, the 12 combinations of three levels of availability
and four rotations are
applied completely at random to 12 paddocks so that each treatment
combination occurs on one paddock.  Also, the three 
levels of availability are assigned
completely at random to 15 animals so that each %treatment
level of availability is assigned to
five animals. The five animals are then grazed together
in sequence on the four paddocks assigned to that
level of availability; the sequence of 4 paddocks is determined by
the order in which the rotations were assigned to them.

The sets for this experiment are paddocks, cattle and treatments %-rotations
and the tiers are ${\cal F}_{\rm paddocks} = \setof{\mbox{Paddocks}}$,
${\cal F}_{\rm cattle\ over\ time} =
                   \setof{\mbox{Animals}, \mbox{Rotations}}$,
and ${\cal F}_{\rm treatments} =
                   \setof{\mbox{Availability}, \mbox{Rotations}}$.
Note that Rotations occurs in the two tiers
${\cal F}_{\rm cattle\ over\ time}$ and ${\cal F}_{\rm treatments}$.
It must be included in the first so that the set of factors in this tier
uniquely indexes the observational units. It occurs in the other tier because
$\mbox{Availability}\wedge\mbox{Rotations}$ is randomized to Paddocks.
Further, it is not possible to have the Rotations in the
Cattle-over-time tier randomized to Paddocks there is no viable unrandomized
design.
There are two randomizations that have to be taken into account in deriving
the analysis:  the randomization of levels of availability
to animals and that of treatments to the paddocks. Note that treatments are
randomized in both randomizations and so this experiment involves double
multiple randomizations: see the randomization diagram in
Figure~\ref{fig:improper}, where the
double randomizations are shown by two arrows which start from the same factor
but go to different tiers. There is no arrow between the two Rotations
in the different tiers because they are the same Rotations.

\begin{figure}[htbp]
\centering
\begin{picture}(7,3)
\put(-1,0.9){\begin{tierbox}3 & Availability\\4 &
Rotations\end{tierbox}}
\put(4.5,0.4){\begin{tierbox}12 & Paddocks\end{tierbox}}
\put(4.5,1.9){\begin{tierbox}15 & Animals\\4 & Rotations\end{tierbox}}
\put(3,0.95){\blob}
\put(3,0.95){\line(-4,-1){0.95}}
\put(2.05,1.3){\vector(3,1){2.85}}
\put(2.05,1.2){\vector(4,-1){2.85}}
\end{picture}
\caption{Double randomizations in the rotational grazing experiment}
\label{fig:improper}
\end{figure}

The structure formulae for the rotational grazing experiment are
\[\begin{array}{c}
  \mbox{15 Animals} \cross \mbox{4 Rotations} \\
  \mbox{12 Paddocks} \pseudo (\mbox{3 Herds} + \mbox{4 Rotations})  \\
  \mbox{3 Availability} \cross \mbox{4 Rotations}.
\end{array}\]
It is necessary to introduce two pseudofactors into the second formula to
account for the association between cattle-over-time entities and paddocks
that arises from the randomization. The first of these recognizes that
certain groups of animals, or herds, were associated with particular
paddocks in the randomization of Availablilty to both Animals and Paddocks.
The second, Rotations, recognizes that different Rotations were associated
with different paddocks when treatments were randomized to paddocks.

The Hasse diagrams corresponding to the first and third formulae are just
standard ones for two crossed factors. That for the second formula
is given Figure~\ref{f:HasseRotate}. The matrices of efficiences are
straightforward with elements that are either 0 or 1.
The analysis of variance that summarizes the relationships between
the elements of the different structures is given in
Table~\ref{tab:ANOVARotateGraze}. It shows how the 11 degrees of freedom
is split between the three terms and that there is no Residual degrees of
freedom for testing Availability differenes and interactions --- hence its
being dubbed improperly replicated.

\begin{figure}[htbp]
\begin{center}
\begin{minipage}{\columnwidth}
\centering
\begin{picture}(4,4)(-2,-1)
\put(0,2){\blob}
\thook{0,2.3}{Universe $1$, $1$}
\put(-1,1){\blob}
\llhook{-1.3,1.25}{\begin{tabular}{@{}c@{}}
     Herds\\3, 2 \end{tabular}}
\put(1,1){\blob}
\rrhook{1.3,1.25}{\begin{tabular}{@{}c@{}}
     Rotations \\4, 3 \end{tabular}}
\put(0,0){\blob}
\bhook{0,-0.3}{\begin{tabular}{@{}c@{}}
     Paddocks \\12, 6 \end{tabular}}
\multiput(0,0)(-1,1){2}{\line(1,1){1}}
\multiput(0,0)(1,1){2}{\line(-1,1){1}}
\end{picture}
\end{minipage}
\end{center}
%\vspace{\baselineskip}
\caption{Hasse diagram for paddocks formula
from the rotational grazing experiment}
%Example~\ref{eg:RotateGraze}}
\label{f:HasseRotate}
\end{figure}

\begin{table}[htbp]
%\renewcommand{\arraystretch}{1.9}
\begin{center}
\begin{tabular}{lrrr}
Source & \multicolumn {3}{c}{DF} \\
\hline
Animals & 14 \\
\quad Paddocks & & 2          \\
\qquad Availability  & & & 2  \\
\quad Residual  & & 12        \\
Rotations & 3 \\
\quad Paddocks & & 3          \\
$\mbox{Animals}\inter \mbox{Rotations}$ & 42 \\
\quad Paddocks & & 6          \\
\qquad $\mbox{Availability} \inter \mbox{Rotations}$  & & & 6  \\
\quad Residual  & & 36       \\
\hline
Total            & 59
\end{tabular}
\end{center}
\caption{Analysis variance table for the rotational grazing experiment}
\label{tab:ANOVARotateGraze}
\end{table}
